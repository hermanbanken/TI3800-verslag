\chapter*{Samenvatting}
\setheader{Samenvatting}

In baansporten is de laatste jaren een ontwikkeling gaande om tijdregistratie te digitaliseren door het gebruik van transponders en detectie-lussen (in de baan). Door deze ontwikkeling zijn nieuwe mogelijkheden ontstaan om ook naast het wedstrijdmoment de sportprestaties in te zien.

Het huidige gebruik van transponders - buiten wedstrijden - is voornamelijk achteraf, terwijl juist tijdens de training zowel sporter als coach het meeste bezig zijn met de prestaties. Het is daarom wenselijk om de resultaten in real-time door te geven aan coaches en sporters zelf. De bestaande oplossingen bieden slechts beperkt inzicht in trainingen, hetgeen de aanleiding voor dit Bachelor Eind Project is geweest.\\

\noindent 
Gedurende dit project is met hulp van de agile ontwikkelmethode SCRUM in combinatie met Team Foundation Server gewerkt aan een applicatie die op een overzichtelijke manier inzicht geeft in trainingen van eigen, maar ook andere sporters. Daarnaast bevat de applicatie een sociale component, waarmee sporters gemotiveerd worden om te blijven/gaan sporten.

Gebruikers kunnen een transponder toekennen aan hun account en groepen aan te maken en/of hier deel in te nemen. Deze groepen bieden hen de mogelijkheid om naast hun eigen resultaten ook andermans resultaten in te zien. Bijvoorbeeld een coach en zijn pupillen. Tevens kunnen zij zich abonneren op andere sporters, handig voor het thuisfront.

Daarnaast bevat dit sociale component ook de mogelijkheid om leaderboards, een soort ranglijst, per verschillende context in te zien. Zo heeft iedere gebruiker zijn eigen records, maar heeft hij ook een positie op de ranglijst van de banen waarop hij gesport heeft en in de groepen waarvan hij deel uitmaakt.\\

\noindent
Om sporters reeds tijdens hun training inzicht te geven in hun prestaties dient alle verkregen data realtime verwerkt en verstuurd te worden naar de mobiele telefoons. Om dit mogelijk te maken draait er in de Microsoft Azure Cloud een server die transponder-doorkomsten verwerkt. De doorkomsten komen binnen vanuit MyLaps, de transponder- en detectielussenleverancier. Vervolgens groeperen we doorkomsten op trainingssessies en ronden, en onderscheiden we de rustperiodes. Het verwerkingsproces steekt vernuftig in elkaar: de beschikbare data wordt steeds een beetje meer verrijkt en steeds naar de applicaties verstuurd.

De applicatie is ontwikkeld met behulp van cross-platform ontwikkel-technieken, waardoor deze relatief eenvoudig uit te brengen is op andere mobiele platformen. De back-end van de applicatie is sport-agnostisch, waardoor deze breed inzetbaar is voor allerlei verschillende soorten baansporten. \\

\noindent 
Uit de enquêtes, die gedurende het project afgenomen zijn onder de doelgroep van de applicatie, is gebleken dat er veel vraag naar een dergelijke applicatie is en de applicatie veel potentie heeft. Ook is er tijdens meetings tussen Emando, de opdrachtgever, en de \ac{KNSB} naar voren gekomen dat de \ac{KNSB} veel interesse heeft in de applicatie en de doorontwikkeling ervan.

Momenteel is de iPhone applicatie klaar om de markt op te gaan, maar uit de enquêtes en vanuit de \ac{KNSB} is naar voren gekomen dat er nog veel functionaliteit is die toegevoegd zou kunnen worden aan de applicatie. Onze aanbeveling is dan ook om de applicatie verder te ontwikkelen en de doelgroep te vergroten door de applicatie ook uit te brengen voor andere platformen.

