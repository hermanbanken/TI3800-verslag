\chapter*{Samenvatting}
\setheader{Samenvatting}

In baansporten is de laatste jaren een ontwikkeling gaande om tijdregistratie te digitaliseren door het gebruik van transponders en detectielussen in de baan. Daardoor zijn nieuwe mogelijkheden ontstaan om ook naast het wedstrijdmoment de sportprestaties in te zien.

Het huidige gebruik van transponders - buiten wedstrijden - is voornamelijk achteraf, terwijl zowel sporter als coach juist \textit{tijdens} de training het meest bezig zijn met de prestaties. Het is daarom wenselijk om de resultaten in realtime door te geven aan coaches en sporters zelf. De bestaande oplossingen bieden slechts beperkt inzicht in trainingen, hetgeen de aanleiding voor dit Bachelor Eindproject is geweest.

\medskip

\noindent 
Gedurende dit project is met hulp van de agile ontwikkelmethode Scrum in combinatie met Team Foundation Server gewerkt aan een applicatie die op een overzichtelijke manier inzicht geeft in trainingen van eigen, maar ook de prestaties van andere sporters. Daarnaast bevat de applicatie een sociale component, waarmee sporters gemotiveerd worden om te blijven of te gaan sporten.

Binnen de applicatie kunnen gebruikers namelijk een transponder toekennen aan hun account en het is mogelijk om groepen aan te maken en/of hier deel in te nemen. De groepen bieden bijvoorbeeld een coach, zijn pupillen en hun thuisfront c.q. aanhang de mogelijkheid om naast de eigen resultaten ook de resultaten van andere groepen in te zien.

Daarnaast bevat dit sociale component ook de mogelijkheid om leaderboards, een soort ranglijst, per verschillende context (baan of groep) in te zien. Zo heeft iedere gebruiker zijn eigen records, maar heeft hij ook een positie op de ranglijst van de banen waarop hij gesport heeft en in de groepen waarvan hij deel uitmaakt.

\medskip

\noindent
Om sporters reeds tijdens hun training inzicht te geven in hun prestaties dient alle verkregen data realtime verwerkt en verstuurd te worden naar de mobiele telefoons. Om dit mogelijk te maken draait er in de Microsoft Azure Cloud een server die doorkomsten van de transponders verwerkt. De doorkomsten komen binnen vanuit \mylaps, de transponder- en detectielussenleverancier. Vervolgens groeperen we doorkomsten op trainingssessies en ronden, en onderscheiden we de rustperiodes. Het verwerken van deze data is bijzonder complex: de beschikbare data wordt in elke stap een beetje meer verrijkt en continu naar de mobiele apps verstuurd, zodat gebruikers altijd de laatst beschikbare informatie zien.

\medskip

\noindent
De applicatie is ontwikkeld met behulp van cross platform ontwikkeltechnieken, waardoor deze relatief eenvoudig uit te brengen is op andere mobiele platformen. De back-end van de applicatie is sport-agnostisch, waardoor deze breed inzetbaar is voor allerlei verschillende soorten baansporten.

\medskip

\noindent 
Uit de enquêtes, die gedurende het project afgenomen zijn onder de doelgroep van de applicatie, is gebleken dat er veel vraag naar een dergelijke applicatie is en dat de applicatie veel potentie heeft. Ook is er tijdens meetings tussen Emando, de opdrachtgever, en de \ac{KNSB} naar voren gekomen dat de \ac{KNSB} veel interesse heeft in de applicatie en de doorontwikkeling ervan.

\medskip

\noindent
Momenteel is de iPhone applicatie klaar om de markt op te gaan, maar uit de enquêtes, gebruikerstest en vanuit de \ac{KNSB} is naar voren gekomen dat er nog veel functionaliteit is die toegevoegd zou kunnen worden aan de applicatie. Onze aanbeveling is dan ook om de applicatie verder te ontwikkelen en de doelgroep te vergroten door de applicatie ook uit te brengen voor andere platformen.