\section{Proces aanpak}

\subsection{Stand up meetings}
Om het overzicht binnen het project te waarborgen, zullen er iedere werkdag aan het begin van de dag stand up meetings met alle projectleden plaatsvinden. Bij deze stand up meetings wordt kort besproken waar momenteel aan gewerkt wordt, in hoeverre dit nog op schema ligt en waar de uitdagingen liggen voor de komende dagen. De opdrachtgever zal minstens twee keer per week aanwezig zijn bij deze meetings, hetgeen naar wens vanzelfsprekend vaker kan.

\subsection{Scrum}
In overleg met de opdrachtgever is er gekozen voor een scrum aanpak voor het project. Scrum is een flexibele manier om software te ontwikkelen. Hierbij gaan we werken in een multidisciplinair team waarmee we in korte sprints, met een vaste lengte van 2 weken, werkende software opleveren en geleidelijk stabiele functionaliteit toe voegen. Een scrum aanpak heeft als voordeel dat na iedere sprint er een werkend product af is, waarna de eisen en doelstellingen gemakkelijk bijgesteld kunnen worden. Met scrum kunnen na de gebruikerstest, halverwege het project, de doelstellingen bijgesteld worden aan de hand van de uitkomst van deze test. Door deze agile methode toe te passen sluit het eindproduct altijd zo goed mogelijk aan op de wensen van de opdrachtgever en de end-users.

\subsection{\acl{tfs}}
Emando gebruikt \ac{tfs} voor het versiebeheer van broncode, het bijhouden van het ontwikkelproces, het tracken van issues en het visualiseren van de voortgang in projecten. De gemakkelijke integratie van \ac{tfs} met Visual Studio zorgt ervoor dat issues en backlog items gemakkelijk toegewezen kunnen worden aan developers, waarna zij deze kunnen openen en hiermee aan de slag kunnen binnen hun IDE. De status van deze items kan zowel vanuit de IDE als online aangepast worden en bij het afvinken van items kunnen deze gekoppeld worden aan versie nummers, zodat deze achteraf makkelijk vindbaar zijn.
Ook zorgt de visualisatie van de zogenaamde ‘burndown’ van het project voor een inzichtelijke manier, waarbij de opdrachtgever in een handomslag de voortgang van het project in kan zien. Daarnaast biedt \ac{tfs} inzicht in de tijd die ieder issue en backlog item kost, hetgeen een goede indicatie is voor de scrum sprint.

\section{Project keuzes}



