Voorafgaand aan het project zijn zowel op proces- als product-niveau een aantal beslissingen genomen. Deze beslissingen hebben een directe invloed gehad op het resultaat en de werkwijze van het project. In dit hoofdstuk worden een aantal van deze keuzes geëvalueerd aan de hand van de bevindingen en resultaten gedurende het project. 

\section{Product}

\subsection{Hoe af is de applicatie?}
Bij de oplevering van het project staat er een werkende applicatie die sporters op een gemakkelijke en overzichtelijke manier inzicht geeft in hun trainingen, maar ook in die van anderen. Daarnaast bevat de applicatie een sociale component die sporters kan motiveren om tot het uiterste te gaan. Deze sociale component zorgt er voor dat sporters de applicatie willen blijven gebruiken. De applicatie berust op een robuuste en uitbreidbare API welke goed gedocumenteerd is. Het is daarmee relatief eenvoudig om extra functionaliteit toe te voegen.

\subsection{Verbeteringen}
Hoewel de applicatie voldoet aan de door ons gedefiniëerde MVP, zijn er enkele verbeterpunten welke toevoegingen kunnen zijn aan de applicatie. Zo zou het mogelijk kunnen zijn om realtime training informatie door te geven aan sporters met behulp van audio cues. 
Verder zou de geaggregeerde data gecombineerd kunnen worden met trainingstijden om gebruikersprofielen te creëren, welke gebruikt kunnen worden om gebruikers suggesties aan te kunnen bieden om met andere sporters van gelijkwaardig niveau te gaan sporten.
Ook is het mogelijk om het sociale component verder uit te breiden naar een achievement systeem. 

Tot slot zou ook het delen van prestaties en records en het exporteren van deze data naar RunKeeper een enorme toevoeging zijn voor de applicatie.

\subsection{Android}
Momenteel is de applicatie slechts beschikbaar voor één mobiel platform: iOS. Naast iOS is ook Android een groot platform waar veel gebruikers van onze doelgroep mee werken, hetgeen de \ac{KNSB} reeds aan heeft gegeven in een van de meetings met Emando. Om het bereik van de applicatie te vergroten, is het aan te raden om de applicatie ook uit te rollen voor Android. Vanwege de cross-platform ontwikkeling met Xamarin, is het slechts nodig om de platform specifieke views te schrijven voor dit platform. Om deze reden kan dit met relatief weinig tijd.

\subsection{Is de applicatie een toevoeging op de reeds bestaande markt?}
Hoewel er zeker ruimte is voor toevoegingen van en verbeteringen aan de huidige applicatie, bevat de applicatie een tal van functionaliteiten welke trainingen voor sporters op een overzichtelijke manier inzichtelijker kunnen maken. Naast het inzien van eigen data, is het ook mogelijk om andermans data in te zien, hetgeen een toevoeging is voor trainingsgroepen en coaches. 
Hoewel deze twee aspecten an sich niet nieuw zijn, is de eenvoud en combinatie er van in één applicatie dat wel. Verder is het sociale component niet terug te vinden in vergelijkbare reeds beschikbare applicaties.

Uit de enquêtes en gebruikerstest is gebleken dat de applicatie veel potentie heeft en er zeker vraag naar is op onder sporters. Hieruit kwam naar voren dat er veel vraag is naar extra functionaliteit (met name de audio-cues), maar de ondervraagden gaven hierbij aan de applicatie met de huidige functionaliteit ook al graag in gebruik te nemen.

\section{Proces}

\subsection{SCRUM overhead}
Omdat de precieze vormgeving van de applicatie naast het MVP niet vast stond en er veel Cutting Edge technieken gebruikt werden met nieuwe mogelijkheden, bleek de flexibele ontwikkelmethode SCRUM een goede oplossing voor ons project. Hiermee was het gedurende het project mogelijk om de doelstellingen en implementatiedetails per sprint bij te stellen. Ondanks het feit dat iedere sprint opnieuw gepland moest worden en dagelijkse standups nodig waren om de projectvoortgang en planning door te spreken, heeft SCRUM veel flexibiliteit en overzicht geboden binnen het project. Het gebuik van SCRUM raden wij dan ook ten zeerste aan bij een volgend (soortgelijk) project.

\subsubsection{Effectiviteit \ac{tfs}}
Bij de planning van het project en het werken in sprints hebben we veelvuldig gebruik gemaakt van \ac{tfs}. Hoewel dit een handige tool is, welke binnen Emando de standaard was bij het plannen van sprints, zit er bij het plannen veel overhead. Omdat iedere functionaliteit opgesplitst moest worden in verantwoordelijkheden, backlog items met verantwoordelijkheden en corresponderende taken met een tijdsbesteding, zat er veel tijd in het op orde houden van de planning en voortgang binnen \ac{tfs}. Ondanks het feit dat Visual Studio goede integratie met \ac{tfs} biedt, dienen taken ook handmatig op orde gehouden te worden om bij de dagelijkse standups een goed beeld van de voortgang van het project te verkrijgen.

Omdat onze opdrachtgever een controlerende rol heeft aangenomen binnen het project, bood \ac{tfs} een uitkomst om op ieder moment gemakkelijk inzicht te verkrijgen in de projectvoortgang en afhankelijkheden. Ook tijdens het project was het voor ons erg prettig om in \ac{tfs} altijd een overzicht te hebben van alle taken en prioriteiten. 

Ondanks het feit dat het inplannen van sprints in \ac{tfs} relatief veel tijd heeft gekost, heeft dit ons in combinatie met de dagelijkse standups veel tijd bespaard doordat het een helder overzicht van de projectvoortgang en afhankelijkheden. Het gebruik van \ac{tfs} sprint planning tools als \ac{tfs} raden wij dan ook ten zeerste aan bij een volgend project.

\subsection{Burndown \& afhankelijkheden}
In \ac{tfs} is het mogelijk om aan de hand van de gemaakte sprint planning een burndown grafiek te genereren, waarin in te zien is hoe de projectvoortgang in een sprint verloopt. Bij het plannen van de sprint dienen daarbij een exact aantal uur per taak aangegeven te worden. Door het gebruik van veel Cutting-Edge technieken die nieuw waren voor ons, hebben sommige taken veel meer en anderen veel minder tijd gekost dan aanvankelijk gedacht. Vanwege de onderlinge afhankelijkheden was het soms lastig om verder te gaan met taken die berustten op andere taken die vertraging op liepen. Omdat deze taken op \ac{tfs} bij een vertraging niet afgevinkt konden worden, bevat de burndown grafiek daarom veel pieken en dalen en wook het merendeel van de dagen de burndown grafiek af van de ideale werklijn. 

Ook bleek bij gedurende sprint regelmatig dat er voor bepaalde taken op de achtergrond extra taken nodig waren om deze naar behoren te laten functioneren. Bij het aanmaken van nieuwe taken binnen \ac{tfs} met corresponderende uren ontstond dan stijgende lijn in de burndown grafiek, omdat er extra uren aan het project toe waren gekend.


\subsection{Verwachtingen verschillende betrokkenen}
Bij het project waren aanvankelijk 3 partijen betrokken: De opdrachtgever, de TU Delft begeleider en de project leden.
Later is door het contact tussen Emando en KNSB ook de KNSB betrokken bij het project. 

Omdat de verwachtingen van alle betrokkenen van afweken, is gezien de tijdsdruk er gekozen voor een middenweg waarin iedereen zich kon vinden. Daarbij is gekozen voor een applicatie die gemakkelijk uitbreidbaar is en voldoet aan alle eisen van MVP (en waar mogelijk meer), voorzien is van een robuuste en stabiele back-end en een goede documentatie van het project en zijn voortgang.  

\subsubsection{Contact met opdrachtgever}
Om goed contact te onderhouden met de opdrachtgever is er voor gekozen om 4 dagen per week op het kantoor van Emando te werken in Amsterdam. Dit maakte het mogelijk om de verwachtingen, project-voortgang en -planning regelmatig af te stemmen in de dagelijkse standups. Ook zorgde dit er voor dat implementatie details en Code Reviews gemakkelijk doorgesproken konden worden, zodat alle implementaties volgens de bedrijfsstructuur van Emando gerealiseerd konden worden.

\subsection{Cutting-edge technieken}
Bij het maken van implementatie keuzes is regelmatig bewust gekozen voor Cutting-edge technieken. Hoewel het gebruik van de nieuwste technologieën veel mogelijkheden met zich meebrengt, kost het gebruiken van deze technieken meestal meer tijd dan het gebruiken van conventionele technieken. Bij het maken van deze afweging was het niet alleen de technische uitdaging, maar met het oog op het toekomstperspectief van de applicatie bleek het verstandig om nieuwe technologieën met veel potentie te gebruiken. Deze technologieën zullen in de toekomst (hoogstwaarschijnlijk) langer ondersteund zullen blijven dan conventionele technieken.
Hierbij zijn alleen technologieën gebruikt die reeds ondersteund werden door de desbetreffende community.

Ondanks het feit dat deze technologieën een steilere learning-curve hebben dan conventionele technieken, boden deze technieken dusdanig meer mogelijkheden dat dit opwoog tegen de benodigde extra tijd.