Voorafgaand aan het project zijn zowel op proces- als productniveau een aantal beslissingen genomen. Deze beslissingen hebben een directe invloed gehad op de werkwijze van het project en het uiteindelijke resultaat. In dit hoofdstuk worden een aantal van deze keuzes geëvalueerd aan de hand van de bevindingen en resultaten gedurende het project. 

\section{Product}

\subsection{Hoe ``af'' is de applicatie?}
Bij de oplevering van het project staat er een werkende applicatie die sporters op een gemakkelijke en overzichtelijke manier inzicht geeft in hun trainingen, maar ook in die van anderen. Daarnaast bevat de applicatie een sociale component die sporters kan motiveren om tot het uiterste te gaan. Deze sociale component zorgt er ook voor dat sporters de applicatie willen blijven gebruiken. De applicatie berust op een robuuste en uitbreidbare API welke goed gedocumenteerd is. Het is daarmee relatief eenvoudig om extra functionaliteit toe te voegen.

\subsection{Verbeteringen}
Hoewel de applicatie voldoet aan de door ons gedefiniëerde MVP, zijn er enkele verbeterpunten welke toevoegingen kunnen zijn aan de applicatie. Zo zou het mogelijk kunnen zijn om realtime trainingsinformatie door te geven aan sporters met behulp van audio cues. 
Verder zou de geaggregeerde data gecombineerd kunnen worden met trainingstijden om gebruikersprofielen te creëren, welke gebruikt kunnen worden om gebruikers suggesties aan te kunnen bieden om met andere sporters van gelijkwaardig niveau te gaan sporten.
Ook is het mogelijk om de sociale component verder uit te breiden naar een achievement-systeem. 

Tot slot zou ook het delen van prestaties en records en het exporteren van deze data naar RunKeeper een toevoeging zijn voor de applicatie.

\subsection{Android}
Momenteel is de applicatie slechts beschikbaar voor één mobiel platform: iOS. Naast iOS is ook Android een groot platform waar veel gebruikers van onze doelgroep mee werken. Dit heeft de \ac{KNSB} reeds aangegeven in een van de meetings met Emando. Om het bereik van de applicatie te vergroten, is het aan te raden om de applicatie ook uit te rollen voor Android. Vanwege de cross-platform ontwikkeling met Xamarin, is het slechts nodig om de platform-specifieke views te schrijven voor Android. Om deze reden kan dit in relatief weinig tijd.

\subsection{Is de applicatie een toevoeging op de reeds bestaande markt?}
Hoewel er zeker ruimte is voor toevoegingen van en verbeteringen aan de huidige applicatie, bevat de applicatie tal van functionaliteiten welke trainingen voor sporters op een overzichtelijke manier inzichtelijker kunnen maken. Naast het inzien van eigen data, is het ook mogelijk om andermans data in te zien, hetgeen een toevoeging is voor trainingsgroepen en coaches. 
Hoewel deze twee aspecten an sich niet nieuw zijn, is de eenvoud en combinatie er van in één applicatie dat wel. Verder is de sociale component niet terug te vinden in vergelijkbare reeds beschikbare applicaties.

Uit de enquêtes en gebruikerstest is gebleken dat de applicatie veel potentie heeft en er zeker vraag naar is onder sporters. Ook kwam naar voren dat er veel vraag is naar extra functionaliteit (met name de audio cues), maar de ondervraagden gaven hierbij aan de applicatie met de huidige functionaliteit ook al graag in gebruik te nemen.

\section{Proces}

\subsection{SCRUM overhead}
Omdat de precieze vormgeving van de applicatie naast het MVP niet vast stond en veel cutting-edge technieken gebruikt werden, bleek de flexibele ontwikkelmethode SCRUM een goede oplossing voor ons project. Hiermee was het gedurende het project mogelijk om de doelstellingen en implementatiedetails per sprint bij te stellen. Ondanks het feit dat iedere sprint opnieuw gepland moest worden en dagelijkse stand-ups nodig waren om de projectvoortgang en planning door te spreken, heeft SCRUM veel flexibiliteit en overzicht geboden binnen het project. Het gebuik van SCRUM raden wij dan ook ten zeerste aan bij een volgend (soortgelijk) project.

\subsubsection{Effectiviteit \ac{tfs}}
Bij de planning van het project en het werken in sprints hebben we veelvuldig gebruik gemaakt van \ac{tfs}. Hoewel dit een handige tool is, welke binnen Emando de standaard was bij het plannen van sprints, zit er bij het plannen veel overhead. Omdat iedere functionaliteit opgesplitst moest worden in verantwoordelijkheden, backlog items met verantwoordelijkheden en corresponderende taken met een tijdsbesteding, zat er veel tijd in het op orde houden van de planning en voortgang binnen \ac{tfs}. Ondanks het feit dat Visual Studio goede integratie met \ac{tfs} biedt, dienen taken ook handmatig op orde gehouden te worden om bij de dagelijkse stand-ups een goed beeld van de voortgang van het project te verkrijgen.

Omdat onze opdrachtgever een controlerende rol heeft aangenomen binnen het project, bood \ac{tfs} een uitkomst om op ieder moment gemakkelijk inzicht te verkrijgen in de projectvoortgang en afhankelijkheden. Ook tijdens het project was het voor ons erg prettig om in \ac{tfs} altijd een overzicht te hebben van alle taken en prioriteiten. 

Ondanks het feit dat het inplannen van sprints in \ac{tfs} relatief veel tijd heeft gekost, heeft het gebruik van TFS ons in combinatie met de dagelijkse stand-ups veel tijd bespaard doordat het een helder overzicht geeft van de projectvoortgang en afhankelijkheden. Het gebruik van \ac{tfs} sprint planning tools als \ac{tfs} raden wij dan ook ten zeerste aan bij een volgend project.

\subsection{Burndown \& afhankelijkheden}
In \ac{tfs} is het mogelijk om aan de hand van de gemaakte sprint planning een burndown-grafiek te genereren, waarin te zien is hoe de projectvoortgang in een sprint verloopt. Bij het plannen van de sprint dienen daarbij een exact aantal uren per taak aangegeven te worden. Door het gebruik van veel cutting-edge technieken die nieuw waren voor ons, hebben sommige taken veel meer en andere veel minder tijd gekost dan aanvankelijk gedacht. Vanwege de onderlinge afhankelijkheden was het soms lastig om verder te gaan met taken die berustten op andere taken die vertraging op liepen. Omdat deze taken op \ac{tfs} bij een vertraging niet afgevinkt konden worden, bevat de burndown-grafiek daarom veel pieken en dalen en week het merendeel van de dagen de burndown-grafiek af van de ideale werklijn. 

Ook bleek bij gedurende sprint regelmatig dat er voor bepaalde taken op de achtergrond extra taken nodig waren om deze naar behoren te laten functioneren. Bij het aanmaken van nieuwe taken binnen \ac{tfs} met corresponderende uren ontstond vanzelfsprekend een stijgende lijn in de burndown-grafiek.


\subsection{Verwachtingen verschillende betrokkenen}
Bij het project waren aanvankelijk drie partijen betrokken: de opdrachtgever, de begeleider vanuit de TU Delft en de projectleden.
Later is door het contact tussen Emando en de KNSB ook de laatste betrokken bij het project. 

Omdat de verwachtingen van alle betrokkenen van elkaar afweken, is gezien de tijdsdruk gekozen voor een middenweg waarin iedereen zich kon vinden. Daarbij is gekozen voor een applicatie die gemakkelijk uitbreidbaar is, voldoet aan alle eisen van MVP (en waar mogelijk meer), voorzien is van een robuuste en stabiele back-end en een goede documentatie van het project en zijn voortgang.  

\subsubsection{Contact met opdrachtgever}
Om goed contact te onderhouden met de opdrachtgever is er voor gekozen om vier dagen per week op het kantoor van Emando in Amsterdam te werken. Dit maakte het mogelijk om de verwachtingen, projectvoortgang en -planning regelmatig af te stemmen in de dagelijkse stand-ups. Ook zorgde dit er voor dat implementatiedetails en code reviews gemakkelijk doorgesproken konden worden, zodat alle implementaties volgens de bedrijfsstructuur van Emando gerealiseerd konden worden.

\subsection{Cutting-edge technieken}
Bij het maken van implementatiekeuzes is regelmatig bewust besloten om cutting-edge technieken te gebruiken. Hoewel het gebruik van de nieuwste technologieën veel mogelijkheden met zich meebrengt, kost het meestal meer tijd dan het gebruiken van conventionele technieken. Dit heeft te maken met de grote hoeveelheid (nog niet gedocumenteerde) mogelijkheden die nieuwe technologieën met zicht meebrengen.

Bij het maken van deze afweging was het niet alleen de technische uitdaging, maar ook het het toekomstperspectief van de applicatie dat de doorslag gaf voor het gebruik van deze technologieën. De weliswaar steilere leercurve heeft ons er niet van weerhouden om deze veelbelovende technieken te gaan gebruiken.