De applicatie is in het kader van het Bachelor Eindproject ``Trainingsapp voor baansporten'' ontwikkeld. Omdat de projectduur van dit project vele malen korter is dan het traject van een volledige applicatie, is de applicatie niet compleet. Zoals beschreven is in de resultaten van de gebruikerstest en de discussie in hoofdstuk~\ref{ch:discussie}, zijn er een tal van functionaliteiten waarmee de applicatie later uitgebreid kan worden. Het is daarom belangrijk dat het project goed opgeleverd wordt, zodat de applicatie door ontwikkeld kan worden.

\section{Oplevering}
Alle code die ontwikkeld is binnen het project is terug te vinden binnen Source Control van het project in \ac{tfs}.
Om eventuele doorontwikkeling van het project te bevorderen bevatten is alle code, waar nodig, voorzien van commentaar. Ook zijn er voor de API's documenten gemaakt waarin hun werk- en gebruikswijze gedocumenteerd is. 

Naast deze documentatie voor de code, zijn er ook een aantal documenten geschreven waarin beschreven staat hoe andere developers aan de slag kunnen met het project. Hierin staat de volgorde van stappen beschreven die ondernomen dient te worden om alle afhankelijkheden en cloud instances draaiende te krijgen.

{\par \bigskip \par \color{red} TODO \par \bigskip \par }

\section{Doorontwikkeling}
Hoewel de applicatie momenteel geschikt is om uit te brengen op de markt, zijn er nog een tal van mogelijkheden om de applicatie door te ontwikkelen. Zoals uit de enquêtes en de meetings met de \ac{KNSB} gebleken is, is er veel vraag naar de applicatie, maar ook naar verdere functionaliteit. 
Enkele van deze gewenste functionaliteiten zijn:

\begin{itemize}
\item Profielen/groepen/trainingsdata onvindbaar maken voor anderen (handig voor topsport(groepen)).
\item Audio-cues, live audio feedback over je training 
\item Ingegratie met hardslagmeter
\item Schatting voor calorieën verbruik
\item Integratie met instelbare trainingsschema's en suggesties hiervoor aan de hand van training resultaten
\item Mogelijkheid om te trainen met GPS, i.p.v. transponders
\end{itemize}

Ook een vergroting van de doelgroep door de applicatie uit te brengen op andere platformen is wenselijk. Binnen \ac{tfs} zijn backlog items aangemaakt voor alle functionaliteit die later wenselijk is, maar nog niet geïmplementeerd is. Deze backlog items zijn gesorteerd op volgorde van prioriteit en kunnen gebruikt worden als leidraad bij de doorontwikkeling van het project.

In het theoretische geval dat er een Azure cloud instance crasht kan het voorkomen dat er aggregaties meerdere malen berekent. Dit resulteert niet in ``verkeerde'' data, maar kan wel zorgen voor dubbele berekeningen over intervallen. Dit probleem is te verhelpen door aggregaties niet te laten werken met een synchronisatie tijdstip, maar in plaats daarvan aggregaties die vertraagd raken de ``te vroeg'' uitgerekende aggregaties te verwijderen en opnieuw uit te laten rekenen. Omdat de kans dat een dergelijke instance crasht bijzonder klein is en het oplossen van dit probleem op technisch vlak een grote hoeveelheid tijd zou kosten is in overleg met Emando er gekozen om in het kader van het Bachelor Eindproject dit voorlopig niet te verhelpen. Dit is een probleem dat in de doorontwikkeling zeker verholpen kan worden.

\section{Toekomst}
De \ac{KNSB} heeft in overleg met Emando besloten de mogelijkheden van huidige applicatie mee te nemen in de nieuwe strategie voor het seizoen 2014-2015, om actieve sporters te benaderen. De functionaliteit van de applicatie past hier zeker goed in, maar het is nog onduidelijk of de huidige applicatie uitgebreid zal worden met functionaliteit en platformen of dat er een complete schaats applicatie komt. De \ac{KNSB} denkt namelijk na over een complete applicatie die naast het trainingsgedeelte ook andere schaatsgerelateerde functionaliteiten bevatten. Deze functionaliteiten zouden onder andere informatie over verenigingen, informatie over (veilig) natuurijs, openingstijden en adressen van schaatsbanen etc. kunnen omvatten.

De back-end van de nieuwe applicatie zal in dit laatste geval berusten op de bestaande back-end die gebouwd is tijdens dit project.