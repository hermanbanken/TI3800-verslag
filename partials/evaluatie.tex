Om er voor te zorgen dat de applicatie op zowel implementatie als functioneel niveau goed in elkaar steekt, hebben we gedurende het project op verscheidene manieren getoetst of de applicatie voldoet aan de vooraf gestelde eisen. De resultaten van deze toetsingen zijn meegenomen in de opvolgende sprints om de applicatie zo goed mogelijk af te stemmen op de wensen van alle betrokken partijen.

\section{KNSB Meetings}
Voorafgaand aan het project is niets afgestemd met de \ac{KNSB}. Gaandeweg het project is er vanuit Emando contact opgenomen met de \ac{KNSB} om hun interesse voor dit project te peilen. Hiervoor bleek enorm veel animo. De \ac{KNSB} heeft aangegeven dat een doorontwikkeling van de applicatie wenselijk is. Ook hebben zij aangegeven dat een Android applicatie wenselijk is, om de doelgroep te vergroten.

\section{Enquête \& gebruikerstest}
Om er zeker van te zijn dat de applicatie aansluit bij de wensen van de doelgroep van onze applicatie, hebben wij een enquête afgenomen om de verwachtingen van de applicatie te peilen vlak na de tweede sprint van het project. Door in een vroeg stadium de verwachtingen van de applicatie te peilen, was het mogelijk om de applicatie gedurende het project aan te passen op deze verwachtingen waar nodig. 

Vlak voor de oplevering van het project hebben we nog een gebruikerstest afgenomen om te kijken hoe de uiteindelijke applicatie hiervan bij de doelgroep bevalt. Hierbij zijn tevens vooraf en achteraf aan de test enquêtes en vragen gesteld aan de gebruikers, om zo een goed beeld te kunnen vormen van hun bevindingen. De verkregen inzichten kunnen gebruikt worden bij de verdere doorontwikkeling en de toekomst van de applicatie. Het testplan en de resultaten zijn te vinden in Appendix~\ref{ch:testplan}

{\par \bigskip \par \color{red} TODO: resultaten van tests bespreken zodra we deze hebben \par \bigskip \par }

\section{\acs{sig} feedback}

De \acf{sig} heeft onze broncode tweemaal geëvalueerd.

{\par \bigskip \par \color{red} TODO: SIG code analyse en resultaten toelichten zodra we deze hebben \par \bigskip \par }