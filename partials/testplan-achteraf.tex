Om het gebruik van de applicatie onder onze doelgroep te bevorderen is het belangrijk om de applicatie aan te laten sluiten bij de verwachtingen en eisen van onze doelgroep. Om deze reden is het uiterst belangrijk dat er in een vroeg stadium gecontroleerd wordt of de functionaliteit die gebouwd wordt, aansluit op de wensen van de gebruikers. Om deze reden is er vlak na onze tweede sprint een gebruikersenquête afgenomen. Met behulp van deze enquête hoopten we inzicht te krijgen in de wensen en verwachtingen van onze doelgroep, maar ook in de reactie van gebruikers bij het zien van de eerste versie van de applicatie.

Aan het eind van het ontwikkelproces vond er een gebruikerstest plaats om te kijken of de functionaliteiten die gebouwd zijn, ook in de praktijk bruikbaar zijn. Aan de hand van de uitkomsten van deze test kunnen er ook beslissingen genomen over de toekomst en eventuele doorontwikkeling van de applicatie.

\subsection{Enquête}
Omdat de enquête in een vroeg stadium van het ontwikkelproces plaats vond vinden, was nog niet alle functionaliteit binnen de applicatie beschikbaar. Bij deze enquête werden daarom screenshots van een eerste versie van de applicatie getoond en ontwerpen van de overige functionaliteit. De gebruikers beantwoordden aan de hand van die beelden de vragenlijst.

In de enquête zelf werden er vragen gesteld om te achterhalen hoe belangrijk/waardevol mensen bepaalde functionaliteiten vinden. Dit werd gedaan worden met behulp van de Likert~Scale~\cite{likert1932technique}.

De enquête is te vinden op pagina \pageref{fig:enquete1} en \pageref{fig:enquete2}. De resultaten van de enquête zijn te vinden op pagina \pageref{sec:enquete-resultaten}.

\subsection{Resultaten enquête}
Er vallen een paar dingen op in de resultaten van de enquête. Allereerst is het marktaandeel van Android erg groot. We hebben er in overleg met de opdrachtgever voor gekozen voor iOS te ontwikkelen, maar de markt voor een Android versie is er dus zeker. 

Daarnaast valt op dat de basis functionaliteit, het tonen van doorkomsten en het inzien van trainingen, erg hoog scoort. De sociale functionaliteit scoort lager, al zijn leaderboards en groepen toch ook erg gewild. Het delen via Facebook en RunKeeper is volgens de respondenten niet belangrijk. Zoals ook uit de overige opmerkingen blijkt is de audio feedback echt een functie die er voor kan zorgen dat de applicatie ook bruikbaar is voor schaaters tijdens de training. Het bekijken van de telefoon tijdens het (hard) schaatsen is waarschijnlijk niet veilig.

Verder merken enkele respondenten op dat het jammer zou zijn als er alleen met transponders gewerkt kan worden. Ze noemen GPS als alternatief, maar uit ervaring weten we dat GPS niet werkt in afgesloten ruimtes zoals een schaatsbaan. Door ondersteuning in te bouwen voor GPS is echter wel de bruikbaarheid uit te breiden naar weg-wielrennen, skeeleren en veel andere sporten. Op de markt van GPS-trackers zijn echter al vele partijen actief, waardoor wij hebben besloten nog geen GPS ondersteuning in te bouwen. Een belangrijke feature die onze applicatie echter wel biedt is het analyseren van rondes, iets wat bij de meeste GPS-trackers geen optie is. In de toekomst kan er dus zeker gekeken worden of GPS geïntegreerd kan worden, ook met het oog op natuurijs en de wensen van de \ac{KNSB}.

\subsection{Gebruikerstest}
Om een echte gebruikerstest te doen, moest er ijs liggen. Gelukkig voor ons werd de verbouwing van Thialf een zomer uitgesteld en lag er zomerijs in Thialf. De baan in Friesland is als enige baan van Nederland gedurende de zomer open en dit jaar lag er ijs vanaf 14 juni. Op 18 juni heeft er een gebruikerstest plaatsgevonden tijdens twee reguliere trainingen. Van 10.15 tot 11.45 was het topsport-uur en daarna was er tot 16.00 gelegenheid voor recreanten om te schaatsen.

Tijdens het topsport-uur hebben we transponders uitgedeeld aan de Activia-ploeg. Annette Gerritsen, Roxanne van Hemert en Laurine van Riessen reden met onze transponders. Daarnaast hebben met hun coach Jac Orie de applicatie doorgenomen. Er was voor team Activia helaas geen tijd om een nabespreking te doen na de training. Wel vonden ze vooraf het idee van de applicatie leuk en waren ze te spreken over de al aanwezige functionaliteit in de applicatie. De functionaliteit om gebruikersprofielen af te schermen werd nog genoemd door een coach van een ander team. Deze functionaliteit hadden wij ook al opgenomen in onze Should-have lijst, maar was nog niet gebouwd.

Na de topsport-training hebben we een grotere groep schaatsers transponders gegeven en heeft Herman ook geschaatst. Voorafgaand aan de test werden de testers kort gebrieft worden over de bedoeling van de test. Gedurende de training van de deelnemers namen we steeds een deelnemer aan de kant om de applicatie en zijn schaats-resultaten te laten zien op een test-telefoon. We hadden op deze telefoons de applicatie al geïnstalleerd en zo ingesteld dat de account gekoppeld was aan de juiste transponder, zodat de gebruikers direct aan de slag konden met de applicatie.

\subsection{Resultaten gebruikerstest}
%%% Algemene conclusie testdag
Zowel de topsporters als recreatieve schaatsers bijzonder te spreken over de applicatie. Velen van de recreanten die wij spraken beschikten (nog) niet over een transponder. Ze gaven aan dat dit kwam door de beperkte mogelijkheden. Het merendeel van hen gaf aan het aanschaffen van een transponder met de komt van de applicatie te heroverwegen. Bij het testen van de applicatie hebben we veel enthousiaste reacties ontvangen. Veel testgebruikers waren verbaasd over de uitgebreide mogelijkheden en het gedetailleerde niveau waarop de applicatie inzicht gaf in hun trainingsdata. Een aantal van hen noemde de \mylaps practice website, waarin zij naar eigen zeggen ``wel eens'' naar hun trainingsdata keken en gaven daarbij aan dat deze velen malen minder inzichtelijk was dan onze applicatie.

Hoewel de respondenten en testers enthousiast waren over de applicatie, bleek uit de test wel dat gebruikers mogelijkheden zagen om de applicatie uit te breiden. De belangrijkste en meest genoemde toevoegingen zijn hieronder genoemd:
\begin{itemize}
\item Profielen/groepen/trainingsdata onvindbaar maken voor anderen (handig voor topsporters).
\item Audio-cues, live audio feedback over je training 
\item Integratie met hartslagmeter
\item Schatting voor calorieën verbruik
\item Integratie met instelbare trainingsschema's en suggesties hiervoor aan de hand van training resultaten
\item Mogelijkheid om te trainen met GPS, i.p.v. transponders
\end{itemize}