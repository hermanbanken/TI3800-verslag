\newcommand{\doelgroep}{}
\label{sec:doelgroep}

\subsection*{Schaatsers}
De voornaamste doelgroep van onze applicatie zijn de recreatieve en amateur schaatsers. Zij moeten met behulp van onze applicatie meer inzicht krijgen in hun prestaties en trainingen. 
Om er voor te zorgen dat deze groep gebruikers onze applicatie verkiest boven de reeds bestaande (minder uitgebreide) systemen, dienen zij snel en gemakkelijk met de applicatie aan de slag te kunnen gaan.

Bij een té ingewikkeld systeem zullen zij afhaken en de overstap naar onze applicatie wellicht niet willen maken.
Omdat zij de belangrijkste doelgroep zijn van de applicatie, is het belangrijk dat zij gemotiveerd worden om de applicatie te blijven gebruiken en hen tevens motiveren om ook anderen aan te sporen dit te doen.
Met behulp van sociale componenten zoals trainingsgroepen, het volgen van andere schaatsers en leaderboards kan onze applicatie hen hierin stimuleren.

\subsection*{Coaches}
Als tweede doelgroep zijn er de coaches van de schaatsers. Dit zijn mensen die de schaatsers begeleiden bij hun trainingen en wedstrijden. Zij beschikken zelf niet altijd over een transponder, omdat niet iedere coach zelf schaatst.
Zij willen graag snel en gemakkelijk inzicht in de data van de schaatsers die zij coachen. Het moet voor hen dus gemakkelijk zijn om snel tussen schaatsers te kunnen switchen en deze prestaties zowel onderling als per training gemakkelijk te kunnen vergelijken.

\subsection*{Overige gebruikers}
Tot slot is er de groep gebruikers die zelf geen deel uitmaakt van de schaatswereld. Zij zijn geen schaatsers of coaches en zijn met name geïnteresseerd in de mogelijkheden van de applicatie en het volgen van kennissen of bekende schaatsers.
Ook zij moeten in staat zijn om de applicatie gebruiken, weliswaar in mindere mate.