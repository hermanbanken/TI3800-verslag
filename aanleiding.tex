In baansporten is de laatste jaren een ontwikkeling gaande om tijdregistratie te digitaliseren door het gebruik van transponders en detectie-lussen in de baan. Door deze ontwikkeling zijn nieuwe mogelijkheden ontstaan om ook naast het wedstrijdmoment de sportprestaties in te zien. Het constateren van de hierna genoemde nieuwe mogelijkheden was de aanleiding van dit project.

Een grote speler op deze markt is MyLaps\footnote{http://www.mylaps.com}. Dit bedrijf installeert en beheert detectie-lussen en is actief bij diverse sporten zoals schaatsen, wielrennen, zwemmen, atletiek en diverse motorsporten. Bij sporten met permanente banen liggen de detectie-lussen het gehele jaar in de baan. Er bestaat de mogelijkheid om op de website van MyLaps doorkomst-tijden in te zien en daardoor wordt er veel getraind met transponder.

Het huidige gebruik van transponders - buiten wedstrijden - is voornamelijk achteraf, terwijl juist tijdens de training zowel sporter als coach het meeste bezig zijn met de prestaties. Het is daarom wenselijk om de resultaten in real-time door te geven aan coaches en sporters zelf.

Op de banen zijn meerdere detectie-lussen geïnstalleerd, terwijl er op de website van MyLaps slechts 1 wordt ontsloten. In Thialf\footnote{Schaatsbaan in Heerenveen, Friesland; van alle Nederlandse banen wordt deze baan het meeste gebruikt voor professionele wedstrijden.} liggen bijvoorbeeld 12 detectie-lussen en op de meeste andere schaatsbanen liggen er tenminste 3. Door de data van meerdere lussen te combineren is een betere indicatie te maken van de snelheid van sporters. Veel trainingselementen bestaan uit korte opdrachten, waar het juist om snelheid gaat. Een enkele lus is dan niet afdoende, omdat de rust voor of na de opdracht mee wordt gewogen. Wanneer er op tactische punten detectie-lussen geinstalleerd zijn is er bijvoorbeeld onderscheid te maken tussen bochten en de rechte stukken, dit gebeurt nu in Thialf al in het professionele circuit.